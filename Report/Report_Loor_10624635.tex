%%%%%%%%%%%%%%%%%%%%%%%%%%%%%%%%%%%%%%%%%
% Journal Article
% LaTeX Template
% Version 1.3 (9/9/13)
%
% This template has been downloaded from:
% http://www.LaTeXTemplates.com
%
% Original author:
% Frits Wenneker (http://www.howtotex.com)
%
% License:
% CC BY-NC-SA 3.0 (http://creativecommons.org/licenses/by-nc-sa/3.0/)
%
%%%%%%%%%%%%%%%%%%%%%%%%%%%%%%%%%%%%%%%%%

%----------------------------------------------------------------------------------------
%	PACKAGES AND OTHER DOCUMENT CONFIGURATIONS
%----------------------------------------------------------------------------------------

\documentclass[twoside]{article}

\usepackage{lipsum} % Package to generate dummy text throughout this template

\usepackage[sc]{mathpazo} % Use the Palatino font
\usepackage[T1]{fontenc} % Use 8-bit encoding that has 256 glyphs
\linespread{1.05} % Line spacing - Palatino needs more space between lines
\usepackage{microtype} % Slightly tweak font spacing for aesthetics

\usepackage[hmarginratio=1:1,top=32mm,columnsep=20pt]{geometry} % Document margins
\usepackage{multicol} % Used for the two-column layout of the document
\usepackage[hang, small,labelfont=bf,up,textfont=it,up]{caption} % Custom captions under/above floats in tables or figures
\usepackage{booktabs} % Horizontal rules in tables
\usepackage{float} % Required for tables and figures in the multi-column environment - they need to be placed in specific locations with the [H] (e.g. \begin{table}[H])
\usepackage{hyperref} % For hyperlinks in the PDF

\usepackage{lettrine} % The lettrine is the first enlarged letter at the beginning of the text
\usepackage{paralist} % Used for the compactitem environment which makes bullet points with less space between them

\usepackage{abstract} % Allows abstract customization
\renewcommand{\abstractnamefont}{\normalfont\bfseries} % Set the "Abstract" text to bold
\renewcommand{\abstracttextfont}{\normalfont\small\itshape} % Set the abstract itself to small italic text

\usepackage{titlesec} % Allows customization of titles
\renewcommand\thesection{\Roman{section}} % Roman numerals for the sections
\renewcommand\thesubsection{\Roman{subsection}} % Roman numerals for subsections
\titleformat{\section}[block]{\large\scshape\centering}{\thesection.}{1em}{} % Change the look of the section titles
\titleformat{\subsection}[block]{\large}{\thesubsection.}{1em}{} % Change the look of the section titles

\usepackage{fancyhdr} % Headers and footers
\pagestyle{fancy} % All pages have headers and footers
\fancyhead{} % Blank out the default header
\fancyfoot{} % Blank out the default footer
\fancyhead[C]{Running title $\bullet$ November 2012 $\bullet$ Vol. XXI, No. 1} % Custom header text
\fancyfoot[RO,LE]{\thepage} % Custom footer text

%----------------------------------------------------------------------------------------
%	TITLE SECTION
%----------------------------------------------------------------------------------------

\title{\vspace{-15mm}\fontsize{24pt}{10pt}\selectfont\textbf{Posture Measurement with Kinect and PCL}} 

\author{
\large
\textsc{Cassandra Loor}\\[2mm]
\normalsize University of Amsterdam \\
\normalsize \href{mailto:john@smith.com}{cassandra.loor@student.uva.nl} 
\vspace{-5mm}
}
\date{}

%----------------------------------------------------------------------------------------

\begin{document}

\maketitle % Insert title

\thispagestyle{fancy} % All pages have headers and footers

%----------------------------------------------------------------------------------------
%	ABSTRACT
%----------------------------------------------------------------------------------------

\begin{abstract}

\noindent This is the abstract

\end{abstract}

%----------------------------------------------------------------------------------------
%	ARTICLE CONTENTS
%----------------------------------------------------------------------------------------

\begin{multicols}{2} % Two-column layout throughout the main article text

\section{Introduction}

\lettrine[nindent=0em,lines=3]{P}osture measurement is the calculation of the angle between limbs. The angles between limbs can be used to evaluate posture in cases of sports or rehabilitation. For example, tennis players can improve their forehand if their arm is fully extended. Another example is where a patient recovers from knee surgery and needs to learn to extend their leg again. To measure posture a 3D-camera is used, combined with the opensource Point Cloud Library (PCL). As PCL is an opensource library, any algorithm developed using this can be applied to different 3D-cameras. For this research, the Microsoft Kinect was used.

%------------------------------------------------

\section{Background}
\subsection{Microsoft Kinect}
The Kinect can be used to gather 3D-images. Microsoft has a its own SDK and apps that can be used to measure posture. The Microsoft apps are very good at tracking a human being, leading to accurate measurements. However, this can only be used with the Kinect. In order to use the same algorithms for multiple types of cameras another approach must be developed.

\subsection{Point Cloud Library}
PCL, A Roitberg 

\subsection{Posture Measurement}
posture measurement

%------------------------------------------------

\section{Problem}
PCL has a demonstration by Alina Roitberg which contains creation and tracking of a human being. The PCL Body Parts Detector module contains the implementation of an SVM that recognizes and tracks body parts. These bodyparts are 3D-pointclouds that are used to determine the location of joints. The existing implementation of Alina Roitberg does this. However, this skeleton implementation is unstable.

\paragraph{Re-estimate body parts}
The Body Parts Detector can mislabel points which leads to smaller or larger body parts. This, in turn, leads to miscalculating the angle between limbs. Therefore, a solution must be found to re-estimate the body parts. This re-estimation must estimate the body parts properly, so that the angle between body parts is represented as the subject in front of the camera.

\paragraph{Handle missing body parts}
The implementation contains room for the estimation of knees and elbows. When these are recognized, the body part point cloud can be used to estimate the location of the joint. When these body parts aren't recognized, the location of joints must be inferred based on the surrounding limbs. These can be the upper arm and forearm or the thigh and lower leg. 

\paragraph{Measure angle between bodyparts}
Finally, the angle between bodyparts must be measured. This is a straightforward problem and can be solved using linear algebra.

Summarized, the following problems need to be solved:

\begin{itemize}
\item Adjust estimation of body parts
\item Properly infer the location of joints
\item Measure angle between body parts
\end{itemize}

%------------------------------------------------

\section{Methods}
There are multiple ways to approach the three problems. The chosen approaches are discussed in this section.

\subsection{Body parts re-estimation}
A method of re-estimation of the clusters is mean shift. As the largest problem of body part inference lies in the extremities, it was chosen to use Mean Shift. ADD BACKGROUND.

\subsection{Estimation of joint location}
LINEAR ALGEBRA EXPLANATION!

\subsection{Angle between body parts}
The angle between body parts can be calculated in two different spaces: 3-dimensional and 2-dimensional. The angle will be calculated in both spaces and represented as a 2D image. Then, based on the full skeleton, it will be determined which approach is most accurate. In order to calculated the angle between the limbs in 3D space, linear algebra is used.

%------------------------------------------------

\section{Results}

\begin{table}[H]
\caption{Example table}
\centering
\begin{tabular}{llr}
\toprule
\multicolumn{2}{c}{Name} \\
\cmidrule(r){1-2}
First name & Last Name & Grade \\
\midrule
John & Doe & $7.5$ \\
Richard & Miles & $2$ \\
\bottomrule
\end{tabular}
\end{table}

\lipsum[5] % Dummy text

\begin{equation}
\label{eq:emc}
e = mc^2
\end{equation}

\lipsum[6] % Dummy text

%------------------------------------------------

\section{Discussion}

\subsection{Subsection One}

\lipsum[7] % Dummy text

\subsection{Subsection Two}

\lipsum[8] % Dummy text

%----------------------------------------------------------------------------------------
%	REFERENCE LIST
%----------------------------------------------------------------------------------------

\begin{thebibliography}{99} % Bibliography - this is intentionally simple in this template

\bibitem[Figueredo and Wolf, 2009]{Figueredo:2009dg}
Figueredo, A.~J. and Wolf, P. S.~A. (2009).
\newblock Assortative pairing and life history strategy - a cross-cultural
  study.
\newblock {\em Human Nature}, 20:317--330.
 
\end{thebibliography}

%----------------------------------------------------------------------------------------

\end{multicols}

\end{document}
